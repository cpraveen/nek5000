\documentclass[12pt, oneside]{article}   	% use "amsart" instead of "article" for AMSLaTeX format
\usepackage{geometry}                		% See geometry.pdf to learn the layout options. There are lots.
\geometry{a4paper}                   		% ... or a4paper or a5paper or ... 
%\geometry{landscape}                		% Activate for for rotated page geometry
%\usepackage[parfill]{parskip}    		% Activate to begin paragraphs with an empty line rather than an indent
\usepackage{graphicx}		
\usepackage{amssymb}

\newcommand{\ud}{\textrm{d}}
\newcommand{\dd}[2]{\frac{\ud #1}{\ud #2}}
\newcommand{\Edt}{E_{\Delta t}}

\title{Eigenvalue computation using time-stepper}
\author{Praveen. C}
%\date{}							% Activate to display a given date or no date

\begin{document}
\maketitle
\section{Eigenvalue problem}
%\subsection{}
Consider the linearized Navier-Stokes (LNS) equation
\[
M_1 \dd{v}{t} = A_1 v + B_1 p, \qquad B_1^\top v = 0
\]
After eliminating the pressure we obtain
\[
\dd{v}{t} = Av
\]
The exact solution of this equation is
\[
v(t) = \exp(A t) v_0 = E(t) v_0
\]
Let $\Edt$ be a numerical scheme to solve the LNS
\[
v_{n+1} = \Edt v_n, \qquad t_{n+1} = t_n + \Delta t
\]
Fix a time $\tau = N\Delta t$ and define
\[
E_\tau = (\Edt)^N
\]
i.e., $E_\tau$ corresponds to performing $N$ time steps of the numerical scheme.

We want to solve the eigenvalue problem
\[
A e = \lambda e
\]
Instead let us consider the eigenvalue problem
\[
E_\tau e = \mu e
\]
The two eigenvalues are related by
\[
\mu = \exp(\lambda \tau)
\]
Hence
\[
\textrm{real}(\lambda) = \frac{1}{\tau} \log|\mu|, \qquad \textrm{imag}(\lambda) = \frac{1}{\tau} \textrm{arg}(\mu)
\]
In the Arnoldi method, the eigenvectors are normalized to unit norm. Since we want to use the $L^2$ norm we will consider the generalized eigenvalue problem
\[
M_1 E_\tau e = \mu M_1 e
\]
where $M_1$ is the mass matrix of velocity. Solving this generalized eigenvalue problem using ARPACK will give us eigenvectors whose $L^2(\Omega)$ norm is unity. Note that we only compute eigenvectors corresponding to velocity.

\section{Plane Poiseuille flow}
Consider a channel of dimensions $[-1,+1] \times [0, 2\pi]$. The base solution has the velocity $(1-y^2, 0)$. We use periodic boundary conditions in the $x$ direction.

\end{document}  